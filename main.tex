\documentclass[a4paper,10pt]{article}
\usepackage[utf8]{inputenc}
\usepackage[italian]{babel}
\usepackage[a4paper]{geometry}
\usepackage[linkbordercolor ={0 1 0}]{hyperref}



\usepackage{amsmath}
\usepackage{amssymb}
\usepackage{amsfonts}
\usepackage{pstricks}

\usepackage{environ}
\usepackage{mdwlist}
 
\title{Programmi ed esercizi per il Senior}
\date{}
%\abstractname{asas}


\topmargin -2cm
\oddsidemargin -1cm
\textwidth 17.5cm
\textheight	25.5cm
\parindent=0mm

\newif\ifsoluzioni 
 \NewEnviron{sol}{\ifsoluzioni \textit{Soluzione:} \BODY\fi} 
 
 \soluzionitrue
 %\soluzionifalse
 %per mostrare le soluzioni, scegliere \soluzionitrue, altrimenti l'altro
 
 
 %questo nuovo environmente serve per evidenziare lo scheletro della lezione 
 \NewEnviron{short}{
 \center
 \parbox{14cm}{\large \BODY \normalsize}
 
 \vspace{0.3cm}
}

\begin{document}
\maketitle


I programmi sono descritti sinteticamente all'inizio, come linee guida. Il testo successivo serve per specificare alcuni dettagli e come controllo per non dimenticare niente per chi fa lezione. 
\bigskip

Le parti contrassegnate con $\bigstar$ sono in più.

\vspace{1cm}
 \section{Programmi Basic}

\subsection{GB - 1 [Metodi Algebrici]}
\begin{short}
 Luoghi di punti con la geometria analitica (classiconi: apollonio, luogo degli ortocentri) e scelte opportune di coordinate; distanze con i prodotti scalari e scrittura di vari punti con i vettori; rette e circonferenze con i complessi (e corde e tangenti); applicazioni della trigonometria.
\end{short}



\subsection{GB - 2 [Trasformazioni]}


\begin{short}
Isometrie: Traslazione, Simmetria, Rotazione. Similitudine. Scrittura di queste trasformazioni in complessi. [$\bigstar$ Affinità] 
Inversione Circolare. Inversione + Simmetria in un triangolo.
\end{short}



\vspace{0.3cm}
\textbf{Isometrie. }Le isometrie sono trasformazioni che conservano la distanza. Le figure mantengono la stessa forma: le rette vanno in rette, circonferenze in circonferenze, poligoni in poligoni, gli angoli mantengono la misura. 

Le isometrie più importanti sono traslazione, riflessione e rotazione. 

La traslazione si definisce con un vettore $\vec{v}$, che manda ogni punto $P$ in $P+\vec{v}$ (in cartesiane e in complessi).

La rotazione si definisce tramite un centro $C$ e un angolo $\alpha$ tra 0 e 360.  In complessi, se il centro è l'origine, il punto z viene mandato in $z\cdot e^{i\alpha}$; se il centro è un altro punto, allora bisogna fare una traslazione, rotazione e traslare indietro: $z\rightarrow (z-c)\cdot e^{i\alpha}+c$.

La riflessione si definisce tramite una retta $r$, ogni punto viene mandato nel suo simmetrico rispetto a questa retta. La riflessione inverte l'orientazione a differenza della traslazione e della rotazione.

Come per la rotazione, per scrivere in complessi la riflessione si compongono tre trasformazioni: si sceglie un punto $c$ sulla retta e sia $\alpha$ l'angolo che forma con l'asse reale, allora $z$ va in $\overline{(z-c)e^{-i\alpha}}\cdot e^{i\alpha}+c=\overline{(z-c)}\cdot e^{2i\alpha}+c$.

\textit{Esempio easy:}  $ABC$ triangolo, $H$ ortocentro, $AH$ interseca $BC$ in $D$ e la circonferenza circoscritta in $N$. Dimostrare $DH=HN$.

\vspace{0.3cm}
[$\bigstar$ Fatti sparsi: 

1) ogni isometria è composizione di al massimo tre riflessioni.

2)Si possono dividere in due gruppi, a seconda se mantengono l'orientamento oppure no. Quelle che mantengono l'orientamento sono traslazione e rotazione, quelle che lo invertono sono riflessione e glissoriflessione(=traslazione lungo una retta e riflessione lungo quella retta), questa sono tutte le isometrie possibili

3) rotazione di $\alpha$+rotazione di $\beta$ = rotazione di $\alpha+\beta$ se $\alpha+\beta \neq 0$, altrimenti è traslazione. Traslazione+rotazione di $\alpha$=rotazione di $\alpha$ con un altro centro. analogamente per rotazione+traslazione]

\vspace{0.4cm}

\textbf{Affinità}

\vspace{0.4cm}

\textbf{Inversione.} A ogni punto $P$ associa $P'$ tale che $OP\cdot OP'=R^2$. Costruzione con le tangenti (per punto esterno) e al contrario per punto interno. È involutiva, scambia interno ed esterno, i punti sulla circonferenza di inversione rimangono gli stessi. 

Le rette per l'origine rimangono rette per l'origine, circonferenze per l'origine diventano rette non per l'origine [questo si può dimostrare], circonferenze non per l'origine diventano circonferenze non per l'origine. Calcolo di $A'B'=\frac{AB\cdot R^2}{OA \cdot OB}$, dire che $OAB$ e $OB'A'$ sono simili. L'inversione conserva gli angoli tra le curve, ma non gli angoli tra punti.

\textit{Esempio} Teorema di Tolomeo.

In complessi, l'inversione nell'origine di raggio $R$ manda $z$ in $R^2\cdot \overline{z}^{-1}$. 


[$\bigstar$ Si può fare un ponte tra potenze e inversione: una circonferenza $\gamma$ è invariata per inversione in $O$ di raggio $R$ se $pow_{\gamma}(O)=R^2$. er esempio $\gamma$ circoscritta ad $ABC$, $P$ è l'intersezione della tangente in $A$ con $BC$, allora l'inversione in $P$ di raggio $PA$ scambia $B$ e $C$ e di conseguenza lascia invariata $\gamma$ ]

\vspace{0.4cm}
Inversione + Simmetria


\subsection{GB - 3 [Sintetica]}
\begin{short}
 Circonferenza di Apollonio. Circonferenza di Feuerbach. Simmediana. Segmenti di tangenza di Incerchi/Excerchi, punti di Gergonne e Nagel. Retta di simson. Applicazioni di potenze e assi radicali. 
\end{short}



\clearpage
\section{Programmi Medium}

\subsection{GM - 1 [Numeri complessi e coordinate baricentriche]}
Numeri complessi nella geometria euclidea. Si assume che si possegga una discreta maneggevolezza con il piano complesso.
Rapido ripasso sulla forma polare dei numeri complessi e significato geometrico delle operazioni.

Condizione di allineamento e scrittura dell'equazione di una retta per due punti. Condizione di parallelismo e scrittura della parallela ad una retta passante per un punto ad essa esterno. Condizione di perpendicolarità e scrittura della perpendicolare ad una retta passante per un punto ad essa esterno. Birapporto fra 4 numeri complessi e condizione di ciclicità.

Equazione di una generica circonferenza. Scelta classica delle coordinate: circonferenza circoscrita $\equiv$ circonferenza unitaria. Punti notevoli nella scelta classica delle coordinate. Esempio di quanto si semplificano i conti: intersezione di due corde generiche. Coordinate $u,v,w$ per l'incentro. 

\vspace{0.5cm}
Definizione di coordinate baricentriche. 

Come verificare l'allineamento di tre punti ed equazione di una retta generica. Intersezione di due rette. Area di un triangolo di cui si conoscono le coordinate dei vertici. Punto all'infinito di una retta. Quando due rette sono parallele?

Punti notevoli e notazione di Conway: baricentro, incentro, ortocentro, circocentro, excentri, nagel, gergonne, lemoine... Coniugati isogonali e coniugati isotomici.

Equazione della circonferenza circoscritta (come coniugato isogonale della retta all'infinito).
Equazione di una circonferenza in posizione generale. Equazione dell'asse radicale fra una circonferenza in posizione generale e la circonferenza circoscritta al triangolo referenziale: relazione di tale equazione con le potenze dei vertici del triangolo referenziale rispetto alla circonferenza in posizione generale. Formula di sdoppiamento per la tangente e la polare.

\clearpage
\subsection{GM - 2, [Geometria proiettiva, poli e polari, quadrilateri armonici]}

\begin{short}
 Lunghezze con segno (velocemente). Birapporto tra 4 punti su una retta. Proiezione del birapporto, quindi birapporto tra 4 rette o 4 punti su circonferenza. Quaterna Armonica. Teorema di Desargues.\\
 Polo e Polare. Teorema di La Hire. Lemma della polare. Teorema di Pappo. Teorema di Pascal. Dualità polo-polare. 
\end{short}

Lunghezze con segno. Definizione di birapporto fra 4 punti $(A,B;C,D)$ su una retta e quando si dicono coniugati armonici. Conservazione del birapporto fra punti per proiezione da un punto esterno. Discussione del caso in cui, proiettando da un punto esterno su una retta, un punto va nel punto all'infinito: cosa diventa il birapporto nel caso in cui un punto sia all'infinito? Unicità del quarto armonico. 
Definizione di birapporto fra 4 rette e relazione con gli angoli che queste formano. Definizione di birapporto fra 4 punti su una circonferenza.

Teorema di Desargues, Teorema di Pappo e Teorema di Pascal.


\vspace{0.5cm}

Definizione proiettiva di polare: data una circonferenza (o due rette) e punto $P$, si traccino le rette che passano per P e secano lo circonferenza (o incontrano le rette) in due punti $A$ e $B$. Il luogo dei punti $X$ tali che $(A,B;P,X)=-1$ è una retta detta polare di $P$ rispetto alla circonferenza (o alle due rette).  Proprietà geometriche nel caso della circonferenza.

Dualità poli-polari. Lemma della polare: Dato un punto $P$ e una circonferenza (o due rette), se traccio due rette secanti che tagliano la circonferenza (o le rette) in due coppie di punti $A,B$ e $C,D$ dimodoché i punti siano nell'ordine $P,A,B$ e $P,C,D$, allora $AD\cap BC$ e $AC\cap BD$ sono sulla polare di $P$ rispetto alla circonferenza (o alle due rette).  Teorema di Brianchion.

Definizione di quadrilatero armonico con i birapporti. In un quadrilatero armonico una diagonale e la tangenti alla circonferenza in cui è inscritto condotte per gli altri due punti concorrono.


\vspace{0.3cm}
\large{\textbf{Versione estesa}}\normalsize

\vspace{0.3cm}
Il setting della geometria proiettiva è quello della retta euclidea a cui viene aggiunto un punto all'infinito, o del piano euclideo in cui viene aggiunta una retta all'infinito.

\vspace{0.3cm}
\textbf{Lunghezze con segno} Su una retta $r$ sono presenti alcuni punti $A,B,C\ldots$. Si scelga un verso sulla retta e si considerino i segmenti su di essa come vettori, con segno positivo se orientati nel verso scelto e negativo altrimenti. Il vantaggio di questo è che vale $\overline{AC}=\overline{AB}+\overline{BC}$ per qualsiasi posizione reciproca di $A,B,C$.

\vspace{0.3cm}
\textbf{Birapporto} Dati 4 punti $A,B,C,D$ su una retta, si definisce il birapporto è la seguente quantità:
$$(A,B;C,D)=\frac{\frac{AC}{AD}}{\frac{BC}{BD}}=\frac{AC\cdot BD}{BC\cdot AD}$$
dove le lunghezze sono prese con segno.

\vspace{0.3cm}
Se $(A,B;C,D)=k$, qual è il valore del birapporto se si permuta l'ordine in cui si prendono i punti? Le $4!=24$ possibilità si dividono in $6$ gruppi in ciascuno dei quali il birapporto è lo stesso. Se si scambiano le due coppie oppure si inverte l'ordine in entrambe il birapporto non cambia: $(A,B;C,D)=(C,D;A,B)=(B,A;D,C)$.

Se si scambiano i primi due o gli ultimi due, il birapporto diventa reciproco: $(A,B;D,C)=(B,A;C,D)=1/k$.

Se si scambia il secondo e il terzo $B \leftrightarrow C$, si ottiene $(A,C;B,D)=1-k$.

Se si scambia il primo e il terzo $A \leftrightarrow C$, si ottiene $(C,B;A,D)=\frac{k}{k-1}$.


Combinando queste trasformazioni, si possono ottenere i valori di $(A,C;D,B)=\frac{1}{1-k}$ e $(A,D;B,C)=\frac{k-1}{k}$.

Un'altra cosa interessante è fissare i punti $A,B,C$ e vedere come varia il birapporto $(A,B;C,D)$ al variare di $D$ sulla retta. Questa è una funzione biettiva dalla retta proeittiva in $\mathbb{R}\cup\infty$, nei casi degeneri in cui $D$ coincide con uno dei punti assume i valori degeneri di $0,1,\infty$; se $D=\infty$, il birapporto vale $AC/BC$.

\vspace{0.3cm}
\textbf{Quaterna Armonica} Quattro punti su una retta si dicono una quaterna armonica se $(A,B;C,D)=-1$. Per quanto detto sulle permutazioni, una quaterna è armonica se e solo se non è degenere e $(A,B;C,D)=(B,A;C,D)$.

Una quaterna armonica dev'essere "incatenata": fissati $A,B$, uno tra $C$ e $D$ deve stare all'interno del segmento $AB$ e uno all'esterno. Analogamente si avrà che uno tra $A$ e $B$ sta all'interno del segmento $CD$ e uno all'esterno.

\vspace{0.3cm}



\clearpage
\subsection{GM - 3 [Configurazione di Miquel, rotomotetia, qualcosa sulle mistilinee e inversioni sintetiche]}

\begin{short}
 Angoli orientati, Miquel su triangolo e su quadrilatero. Lemma della rotomotetia. Quadrilatero completo e rotomotetie presenti nella configurazione. Altre applicazioni di inversione. mistilinei,
\end{short}

Angoli orientati ed esercizi/complementi sui quadrilateri ciclici. 

Punto di Miquel riferito a una terna di punti presi sui lati di un triangolo. Punto di Miquel riferito a un quadrilatero. Facendo opportuno riferimento all'esercizio 2 della sezione \textbf{GM-1}, osservare che il punto di Miquel di un quadrilatero $ABCD$ è il centro della \emph{spilar similarity} che manda $AB$ in $DC$ o $AD$ in $BC$. Il quadrilatero $ABCD$ è ciclico se e solo se il punto di Miquel $M$ sta su $QR$, dove $Q=AB\cap CD$ e $R=AD\cap BC$.

Nel caso di ciclicità:
\begin{itemize}
	\item $OM$ è perpendicolare a $QR$, essendo $O$ il circocentro di $ABCD$;
	\item $A,C,M,O$ e $B,D,M,O$ sono conciclici;
	\item $AC$, $BD$ e $OM$ sono concorrenti in $P$;
	\item $MO$ biseca $\angle CMA$ e $\angle BMD$;
	\item $P$ e $M$ sono inversi rispetto alla circonferenza circoscritta al quadrilatero $ABCD$.
\end{itemize} 

Un'avventura mistilinea: considerati quattro punti in senso antiorario su una circonferenza $\Gamma$ ($A,B,C,D$) ed essendo $P=AC\cap BD$, considero $\omega$ tangente ai segmenti $AP$ e $BP$ e a $\Omega$ rispettivamente in $E$, $F$ e $T$. Provare le seguenti:
\begin{itemize}
	\item $TE$ biseca l'arco $AC$ che contiene $D$;
	\item Detto $I$ l'incentro di $ABC$, $IFTB$ è ciclico e $I\in EF$
	\item Detto $J$ l'incentro di $APB$ allora $TJFB$ è ciclico e $TJ$ biseca $\angle ATB$.
\end{itemize}

Ripasso delle proprietà base riguardanti l'inversione. $\sqrt{bc}$-inversione più simmetria: risoluzione di alcuni problemi.
\clearpage

 









 \clearpage
 
 
\section{Esercizi Basic}
\subsection{GB-1, Esercizi}
\begin{enumerate}
\item Teoremi di Napoleone e Vecten (enunciato in G2)

\end{enumerate}


\subsection{GB-2, Esercizi}
\begin{enumerate}
       \item Fare i conti per traslazioni, rotazioni, riflessioni, inversione in complessi. 

       \item \textbf{Problema di Fagnano} Sia $ABC$ un triangolo acutangolo, $P,Q,R$ tre punti variabili sui lati $BC,AC,AB$ rispettivamente. Per quale posizione dei tre punti il perimetro del triangolo $PQR$ è minimo?
       
      \begin{sol}Sia $P_1$ il simmetrico di $P$ rispetto $AB$ e $P_2$ rispetto $AC$. Allora il perimetro $PR+RQ+QP=P_1R+RQ+QP_2$ è la lunghezza della spezzata $P_1RQP_2$, fissato P è minimo se i quattro punti sono allineati. Inoltre $\widehat{P_1AP_2}=2\cdot\widehat{BAC}$ e $AP_1=AP_2=AP$, quindi $P_1P_2=AP \sin{\widehat{BAC}}$ è minimo quando è minimo $AP$. Quindi $P$ è piede dell'altezza da $A$, e in tale caso anche $Q$ e $R$ lo sono
       \end{sol}
      
       \item \textbf{Teorema di Napoleone} Sia $ABC$ un triangolo e si costruisca un triangolo equilatero su ciascuno dei lati di $ABC$, esterno ad esso. Chiamati $O_A,O_B,O_C$ i centri dei tre triangoli, dimostrare che $O_AO_BO_C$ è un triangolo equilatero.
     
%       \begin{sol}Siano $A_1,B_1,C_1$ i vertici dei triangoli equilateri. $BA=\sqrt{3}BO_C$ e analogamente per gli altri lati. Una rotazione di 30 centrata in $B$ manda $BO_C$ in $BA$ e $BO_A$ in $BA_1$. Il triangolo $BO'_CO_A'$ è simile a $BAA_1$, quindi per Talete $O_CO_A=O'_CO'_A=\frac{1}{\sqrt{3}}AA_1=O_BO_C$. Quindi i tre lati sono uguali.
%      
%       Si fa benissimo in complessi (per G1). Su cut-the-knot ci sono tanti approcci https://www.cut-the-knot.org/proofs/napoleon_complex2.shtml . Si può 
%       \end{sol}
%             
%       \item \textbf{Teorema di Vecten } Sia $ABC$ un triangolo e si costruisca un quadrato su ciascuno dei lati di $ABC$, esterno ad esso. Chiamati $O_A,O_B,O_C$ i centri dei tre quadrati, dimostrare che $O_AO_BO_C$ è un triangolo equilatero.
%       
%       \item \textbf{Eserciziario Senior 17, G2 -10} Siano $ABMN$ e $BCQP$ i quadrati costruiti sui lati $AB$ e $BC$ di un triangolo, esternamente al triangolo stesso.
%       Dimostrare che i centri di tali quadrati ed i punti medi di $AC$ e $MP$ sono i vertici di un quadrato.
%       
%       \begin{sol}sia $L$ il centro di $ABMN$ e $R$ di $BCQP$, $J$ il punto medio di $AC$. $LJ$ è parallelo a $MC$ per omotetia di centro $B$ e fattore 2, inoltre dopo una rotazione di $90^{\circ}$ va in $BP$ che è parallelo a $JR$. Da questo si deduce che $LJ=JR$ e sono ortogonali. Analogamente si fa per il punto medio di $MP$
%       \end{sol}
%       
%       \item \textbf{Teorema di Tolomeo} Sia $ABCD$ un quadrilatero, $AC\cdot BD\leq AD\cdot BC + AB\cdot CD$ e l'uguale vale sse $ABCD$ è ciclico.
%     
%     
% 	 \item  \textbf{[Lemma della \textit{simmediana}]} Sia $ABC$ un triangolo inscritto in una circonferenza $\gamma$. Le tangenti a $\gamma$ in $B$ e $C$ si intersecano in $P$.
%     
%      Mostrare che $AP$ è \textit{simmediana} relativa a $BC$, \textit{i.e.} simmetrica della mediana relativa a $BC$ rispetto alla bisettrice dell'angolo $\angle BAC$.    
%      
%      \item \textbf{IMOSL2013 - G2}  Sia $ABC$ un triangolo, e sia $\omega$ la sua circonferenza circoscritta.  Siano $M$ il punto medio di $AB$, $N$ il punto medio di $AC$, $T$ il punto medio dell’arco $BC$ di $\omega$ che noncontiene $A$. La circonferenza circoscritta al triangolo $AMT$ interseca l’asse di $AC$ in un punto $X$ interno al triangolo $ABC$. La circonferenza circoscritta al triangolo $ANT$ interseca l’asse di $AB$ in un punto $Y$ interno al triangolo $ABC$. Le rette $MN$ e $XY$ si intersecano in $K$ .
%      
%      Dimostrare che $KA=KT$.
%      
%      \begin{sol}La simmetria rispetto all'asse di $AT$ manda $M$ in $X$ e $N$ in $Y$, quindi $K$ rimane fisso e sta sull'asse.
%      \end{sol}
%      
%      \item \textbf{EGMO 2016 - 4} Due circonferenze aventi lo stesso raggio, $\omega_1$ e $\omega_2$ , si intersecano in due punti distinti $X_1$ and $X_2$ . Si consideri una circonferenza $\omega$ tangente esternamente a $\omega_1$ nel punto $T_1$ e internamente a $\omega_2$ nel punto $T_2$ . Si dimostri che il punto d’intersezione fra le rette $X_1T_1$ e $X_2T_2$ giace su $\omega$.
%       \begin{sol} Inversione in $X_1$
%       \end{sol}
%     
% 	\item \textbf{IMOSL2011 - G4} Sia $ABC$ un triangolo acutangolo e $\Gamma$ la sua circonferenza circoscritta. Sia $B_0$ il punto medio di $AC$ e $C_0$ il punto medio di $AB$. Sia $D$ il piede dell'altezza da $A$ su $BC$ e sia $G$ il baricentro di $ABC$. Sia $\omega$ la circonferenza passante per $B_0,C_0$ e tangente a $\Gamma$ in un punto $X\neq A$. 
% 	
% 	Dimostrare che $D,X,G$ sono allineati.
% 	
% 	\begin{sol}
% 	\emph{nota: la soluzione proposta è basic difficile/medium facile}
% 	
% 	Inversione + simmetria di centro A e raggio $\sqrt(AB*AB_0)$, scambia B e $B_0$, C e $C_0$, manda $D$ nel centro di $\Gamma$ O e $\omega$ in una circonferenza per $B$ e $C$ tangente all'immagine di $\Gamma$, $B_0C_0$, in un punto $Y$. Poiché $BC$ e $B_0C_0$ sono paralleli, $Y$ sta sull'asse di $BC$, quindi $OY$ è perpendicolare a $B_0C_0$. 
% 	
% 	Sia $T$ l'intersezione delle tangenti a $\Gamma$ per A,X e di $B_0C_0$, è centro radicale di $\Gamma, \omega$ e la circoscritta a $AB_0C_0$. ATXYO è ciclico, l'immagine sotto inversione è la retta XDY. Ora basta mostrare DY intersecato la mediana $AA_0$ è G, ma $AD$ è il doppio di $XA_0$ e sono paralleli, quindi l'intersezione è proprio G.
% 	\end{sol}
% 
% 	
% 	 \item \textbf{[Teorema di Feuerbach]} Mostrare che la circonferenza di Feuerbach è tangente alla circonferenza inscritta e alle circonferenze exinscritte.
%     
%     \begin{sol} Sia $M$ il punto medio di $BC$ e $D$ e $G$ rispettivamente i punti in cui la circonferenza inscritta e quella ex-inscritta opposta ad $A$ incontrano $BC$. Invertire in $M$ con raggio $MD$.
% 
%     Per prima cosa si nota che il piede della perpendicolare e il piede della bisettrice su BC si scambiano perché MH\cdot MI=MD^2. Inoltre si mostra passando per la circoscritta che la retta immagine della circonferenza dei nove punti fa un angolo di beta - gamma con BC. Dunque la cfr dei nove punti va nella simmetrica della retta BC rispetto alla bisettrice che tange entrambe le circonferenze inscritta ed exinscritta. Inoltre queste due si scambiano
%     \end{sol}
 \end{enumerate}


\subsection{GB-3, Esercizi}
\begin{enumerate}

	\item \emph{[Copiato da GM]} Sia $ABC$ un triangolo con ortocentro $H$ e siano $D$, $E$ e $F$ i piedi delle altezze che cadono sui lati $BC$, $CA$ e $AB$ rispettivamente. Sia $T=EF\cap BC$.
	
	Mostrare che $TH$ è perpendicolare alla mediana condotta da $A$.
	
	\begin{sol}Oltre alla soluzione per inversione, pensavo anche qualcosa con gli assi radicali: il centro radicale delle circonferenze per $AEFH, BCH, ABEF$ è T, quindi TH passa per l'intersezione di AEFH e BCH che chiamo P. Allora APH è retto in quanto diametro


    inversione di centro A e raggio $AH\cdot HD$. La tesi diventa equivalente a mostrare che la circonferenza per D (immagine di H), per l'intersezione della circoscritta con AEF (immagine di T) e A ha la retta AM come diametro. Questo segue perché in effetti $M,T',A,D$ sono ciclici
	\end{sol}
	
	
	\item Sia $ABC$ un triangolo, $E,F$ i piedi delle altezze su $AC,AB$. Sia $H$ l'ortocentro, $M$ il punto medio di $BC$ e $Q$ l'intersezione più vicina ad $A$ di $HM$ con la circoscritta $\Gamma$. Sia $T=EF\cap BC$. Dimostrare che $T,Q,A$ sono allineati.
	
	\begin{sol} Usare due fatti 1) Il punto Q è l'intersezione di $\Gamma$ con la circonferenza di diametro $AH$ ed è allineato con H,M e il simmetrico di A rispetto O 2) Assi radicali di AQH, ABC, BCEF.
\end{sol}	
	
	\item Sia $ABC$ un triangolo con $I$ incentro e $I_A$ centro della circonferenza ex-inscritta relativa ad $A$. Sia $\Gamma$ la circonferenza circoscritta ad $ABC$ e sia $M$ il punto medio dell'arco $BC$ non contenente $A$.
	
	Dimostrare che $B,I,C,I_A$ si trovano su una stessa circonferenza di centro $M$
	
	\item Sia $ABC$ un triangolo, $\omega$ la circonferenza inscritta tangente a $BC$ in $D$. Sia $M$ il punto medio di $BC$ e $E$ il simmetrico di $D$ rispetto a $M$. Sia $T$ il diametralmente opposto a $D$ in $\omega$. Dimostrare che $A,T,E$ sono allineati.
	
\begin{sol}Calcolare i segmenti di tangenza di inscritta ed ex-inscritta, poi omotetia in $A$
 
\end{sol}

	
	\item Proprietà varie della circonferenza di Feuerbach. 
	\end{enumerate}
\clearpage

\section{Esercizi Medium}
\subsection{GM - 1, Esercizi}
\begin{enumerate}
	\item \textbf{[Scrittura del coniugato isogonale in complessi]} Dimostrare che in un triangolo $abc$ inscritto in una
	circonferenza unitaria centrata nell'origine, il coniugato isogonale di $p$ è
	\begin{equation}
	q=\frac{-p+a+b+c-\overline{p}(ab+bc+ca)+\overline{p}^2abc}{(1-p\overline{p})}.
	\end{equation}
	\item \textbf{[Seconda intersezione di due circonferenze in complessi]} Siano dati 4 punti $a, b, c, d$ nel piano complesso che non formano un parallelogrammo.
	
	Mostrare che esiste una e una sola rotomotetia che
	manda $a$ in $b$ e $c$ in $d$. Detto $x$ il centro di tale rotomotetia e $\alpha$ la ragione, si ha
	$$
	c=\frac{ad-bc}{a-b-c+d}
	$$
	$$
	\alpha=\frac{b-d}{a-c}.
	$$
	
	Mostrare che l'intersezione delle circonferenze circoscritte a $ABX$ e $CDX$ dove $AC$ e $BD$ sono segmenti non paralleli le cui rette si intersecano in $X$, è il centro della rotomotetia che manda $A$ in $B$ e $C$ in $D$. 
	\item \textbf{[Una caratterizzazione della \textit{polare} come luogo dei \emph{quarti armonici}]} Sia $\gamma$ la circonferenza unitaria centrata nell'origine e sia $P$ un punto qualsiasi. Siano $r$ ed $s$ la polare di $P$ rispetto a $\Gamma$ e una retta passante per $P$ rispettivamente. 
		\begin{itemize}
		\item Mostrare che $r$ ha equazione
			\begin{equation}
			x\bar{p}-2+\bar{x}p=0
			\end{equation}
			dove $p$ è il numero complesso associato a $P$.
		\item  Supponiamo che $s$ intersechi $\gamma$ in $A_1, A_2$, ed $r$ in $Q$. Mostrare che $(P,Q;A1,A2)=-1$.
		\end{itemize}
	\item \textbf{[Scrittura del circocentro di un triangolo generico in complessi]} Mostrare che il circocentro del triangolo $z_1z_2z_3$ è
		\begin{equation}
		\frac{z_1\bar{z_1}(z_2-z_3)+z_2\bar{z_2}(z_3-z_1)+z_3\bar{z_3}(z_1-z_2)}{\begin{vmatrix}
			z_1 & \bar{z_1} & 1 \\
			z_2 & \bar{z_2} & 1 \\
			z_3 & \bar{z_3} & 1 
			\end{vmatrix}}.
		\end{equation}
	\item \textbf{[Teorema di Brocard]} Sia $ABCD$ un quadrilatero inscritto in una circonferenza di centro $O$. Le rette $AB$ e $CD$ si intersecano in $E$, le rette $AD$ e $BC$ si intersecano in $F$ e le rette $AC$ e $BD$ si intersecano in $G$. 
	
	Mostrare che $O$ è ortocentro di $EFG$.
	\item Sia $ABC$ un triangolo di ortocentro $H$. Da $A$ si conducano le due tangenti alla circonferenza di diametro $BC$ che la intersecano in $P$ e $Q$. 
	
	Mostrare che $H\in PQ$.
	%Hint: la circonferenza unitaria è quella di diametro BC.
	%I punti x che stanno su tale circonferenza e per cui ax %\perp ox soddisfano una quadratica. Sia h' l'intersezione %di ah con pq. Basta mostrare che ch' \perp ab. 
	\item \textbf{[Una caratterizzazione del \textit{punto di Lemoine}]} Sia $ABC$ un triangolo e siano $D$, $E$ e $F$ i punti medi di $BC$, $CA$ e $AB$ rispettivamente. Siano $X$, $Y$ e $Z$ i punti medi delle altezze condotte da $A$, $B$ e $C$ rispettivamente. 
	
	Mostrare che $DX$, $EY$ e $FZ$ si intersecano in un punto di coordinate baricentriche $[a^2:b^2:c^2]$. Chi è tale punto nel triangolo referenziale?
	
	$(\star)$ Mostrare che tale punto (il \textit{punto di Lemoine}) è l'unico punto ad essere baricentro del proprio triangolo pedale.
	\item \textbf{[Coordinate dei vertici del \textit{triangolo tangenziale} in baricentriche]}  Dato un triangolo $ABC$ referenziale in un sistema di coordinate baricentriche, mostrare che la tangente condotta da $A$ alla circonferenza circoscritta ad $ABC$ ha equazione
	\begin{equation}
	c^2y+b^2z=0.
	\end{equation}
	
	Ciclando opportunamente, calcolare le coordinate dei vertici del triangolo tangenziale (\textit{i.e.} il triangolo formato dalle intersezione delle tangenti condotte da $A$, $B$ e $C$ alla circonferenza circoscritta ad $ABC$).
	\item Sia dato un triangolo $ABC$ e un punto $P$ di coordinate baricentriche $[u:v:w]$ scegliendo come triangolo referenziale $ABC$.
		\begin{itemize}
			\item \textbf{[Proiezione di un punto sui lati in baricentriche]} Mostra che, dette $P_A$, $P_B$ e $P_C$ le proiezioni di $P$ sui lati $BC$, $CA$ e $AB$, si ottiene
			$$
			P_A = [0: S_Cu+a^2v:S_Bu+a^2w]
			$$
			$$
			P_B = [S_Cv+b^2u: 0: S_Av+b^2w]
			$$
			$$
			P_C = [S_Bw+c^2u: S_Aw+c^2v: 0]
			$$			
			dove $S_A=\displaystyle\frac{b^2+c^2-a^2}{2}$ e cicliche.
			\item  \textbf{[Punto all'infinito della retta perpendicolare in baricentriche]} Usando il punto precedente mostrare che 
			il punto all'infinito di una retta perpendicolare a $px+qy+rz=0$ è
			\begin{equation}
			[S_Bg-S_Ch:S_Ch-S_Af:S_Af-S_Bg]
			\end{equation}
			dove $[f:g:h]=[q-r:r-p:p-q]$ è il punto all'infinito della retta.
		\end{itemize}
	\item \textbf{[Intersezione delle ceviane per un punto P con la circoscritta in baricentriche]} Sia $P=[u:v:w]$, dove le coordinate baricentriche 
	sono riferite ad $ABC$. Dette $P^A$, $P^B$ e $P^C$ rispettivamente le intersezioni di $AP$, $BP$ e $CP$ con la circonferenza circoscritta, mostrare che 
	$$
	P^A=\left[\displaystyle\frac{-a^2vw}{c^2v+b^2w}:v:w\right]
	$$
	$$
	P^B=\left[u:\displaystyle\frac{-b^2uw}{a^2w+c^2u}:w\right]
	$$
	$$
	P^C=\left[u:v:\displaystyle\frac{-c^2uv}{a^2v+b^2u}\right].
	$$
	\item Ricordiamo il seguente fatto noto di geometria elementare: un punto $P$ sta sulla circonferenza circoscritta ad un triangolo $ABC$ se e solo se le sue proiezioni sui lati $AB$, $BC$ e $CA$ sono allineate (su quella che si chiama \textit{retta di Simson}). 
	
	Usando questo fatto e l'esercizio 9 mostrare che l'equazione della circonferenza circoscritta al triangolo referenziale è
	\begin{equation}
	a^2yz+b^2xz+c^2xy=0.
	\end{equation}
	\item Mostrare che l'asse radicale fra la circonferenza circoscritta al triangolo referenziale e 
		\begin{itemize}
			\item la circonferenza di Feuerbach è $S_Ax+S_By+S_Cz=0$.
			\item la circonferenza inscritta è $(p-a)^2x+(p-b)^2y+(p-c)^2z=0$, essendo $p=\displaystyle\frac{a+b+c}{2}$.
		\end{itemize} 
	\item \textbf{[Distanza fra due punti in baricentriche]} Siano $P=[u:v:w]$ e $Q=[u':v':w']$ le coordinate \textbf{baricentriche esatte} di due punti rispetto a un triangolo referenziale $ABC$.
	\begin{itemize} 
	\item Mostrare che
	\begin{equation}
	PQ^2=S_A(u-u')^2+S_B(v-v')^2+S_C(w-w')^2.
	\end{equation}
	\item Dato un generico punto $P=[u:v:w]$, mostrare che 
	\begin{equation}
	AP^2=\frac{c^2v^2+2S_Avw+b^2w^2}{(u+v+w)^2}
	\end{equation}
	e dedurre, ciclicamente, le espressioni per $BP^2$ e $CP^2$.
	\end{itemize}
\end{enumerate}
\clearpage
\subsection{GM - 2, Esercizi}
\begin{enumerate}
	\item Siano $\gamma_1$ e $\gamma_2$ due circonferenze di centri $O_1$ e $O_2$ rispettivamente. Siano $S_1$ e $S_2$ rispettivamente il centro di similitudine interno ed esterno di $\gamma_1$ e $\gamma_2$.
	
	Mostrare che $(O_1,O_2;S_1,S_2)=-1$.
	\item Siano $A$, $C$, $B$ e $D$ allineati in quest'ordine su una retta. Siano $M$ e $N$ i punti medi dei segmenti
	$CD$ e $AB$ rispettivamente. 
	
	Mostrare che sono equivalenti le seguenti proprietà:
	\begin{itemize}
		\item $(A,B;C,D)=-1$;
		\item $\displaystyle\frac{2}{AB}=\displaystyle\frac{1}{AC}+
		\displaystyle\frac{1}{AD}$;
		\item $MA\cdot MB=MC^2$;
		\item $CA\cdot CB=CD\cdot CN$;
		\item $AB^2+CD^2=4MN^2$.
	\end{itemize}
	\item Siano $\gamma_1$ e $\gamma_2$ due circonferenze \textit{ortogonali} di centri $O_1$ e $O_2$ rispettivamente. Una generica retta passante per $O_1$ interseca $\gamma_1$ in $A$ e $B$ e interseca $\gamma_2$ in $C$ e $D$.
	
	Mostrare che $(A,B;C,D)=-1$.
	\item \textbf{[Conservazione del birapporto per inversione]} Assumiamo che $A$, $B$, $C$ e $D$ siano allineati o conciclici. Siano $A'$, $B'$, $C'$ e $D'$ (allineati o conciclici) le immagini dei precedenti punti tramite un'inversione circolare di centro $O\notin\left\{A,B,C,D\right\}$  qualsiasi. Allora
	\begin{equation}
	(A,B;C,D)=(A',B';C',D').
	\end{equation}
	
	Cosa succede se $O\in \{A,B,C,D\}$?
	%\item \textbf{[Unicità del quarto armonico]}
	%Assumiamo che $A$, $B$, $C$, $D_1$ e $D_2$ %siano conciclici o allineati.
	%
	%Mostrare che se $(A,B;C,D_1)=(A,B;C,D_2)$ %allora $D_1 \equiv D_2$.
    \item Sia $ABC$ un triangolo e $M$ un punto sul segmento $BC$. Sia $N$ preso sulla retta di $BC$ dimodoché $\angle MAN=90$.
    
    Mostrare che $(B,C;M,N)=-1$ se e solo se $AM$ è bisettrice dell'angolo $\angle{BAC}$.
    \item Sia $ABC$ un triangolo scaleno e sia $D \in AC$ tale che $BD$ è la bisettrice di $\angle ABC$.
    Siano $E$ ed $F$ i piedi delle perpendicolari tracciate rispettivamente da $A$ e da $C$ sulla retta $BD$ e
    sia $M \in BC$ tale che $DM \perp BC$.
    
    Mostrare che $\angle EMD=\angle DMF$.
    \item \textbf{[Teorema della farfalla]} Sia $MN$ una corda di una circonferenza $\gamma$ e sia $P$ il suo punto medio. Siano $AB$ e $CD$ due corde qualsiasi di $\gamma$ che si intersecano in $P$ dimodoché $A$ e $C$ siano nello stesso semipiano generato dalla retta su cui giace $MN$. 
    
    Mostrare che $AD$ e $BC$ intersecano la corda $MN$ in due punti equidistanti da $P$. 
    \item Sia $ABCD$ un quadrilatero circoscritto a una circonferenza e siano $M$, $N$, $P$ e $Q$ i punti di tangenza di $AB$, $BC$, $CD$ e $DA$ con la circonferenza rispettivamente. 
    
    Mostrare che $AC$, $BD$, $MP$ e $NQ$ sono concorrenti.
    \item \emph{[Copiato in GB]} \textbf{[Lemma della \textit{simmediana}]} Sia $ABC$ un triangolo inscritto in una circonferenza $\gamma$. Le tangenti a $\gamma$ in $B$ e $C$ si intersecano in $P$.
    
    Mostrare che $AP$ è \textit{simmediana} relativa a $BC$, \textit{i.e.} simmetrica della mediana relativa a $BC$ rispetto alla bisettrice dell'angolo $\angle BAC$.
    
    \item Sia $ABCD$ un quadrilatero ciclico. Le rette $AB$ e $CD$ si intersecano in un punto $E$ e le diagonali $AC$ e $BD$ si intersecano in un punto $F$. Sia $H$ l'intersezione delle circonferenze circoscritte ai triangoli $AFD$ e $BFC$. 
    
    Mostrare che $\angle EHF=90^{\circ}$.
    
    \item Sia $ABCD$ un quadrilatero armonico inscritto in una circonferenza $\gamma$ di centro $O$ con diagonali $AB$ e $CD$. Sia $M$ il punto medio di $AB$.
    
    Mostrare $MA$ è la bisettrice dell'angolo $\angle CMD$.
    \item Usando gli argomenti della lezione \textbf{G2 - Medium} mostrare il \textbf{Teorema di Brocard} contenuto nella raccolta degli esercizi relativi alla lezione \textbf{G1 - Medium}. 
    
    \item Sia $\omega$ la circonferenza inscritta in un triangolo $ABC$ e sia $I$ il suo centro. $\omega$ interseca $BC$, $CA$ e $AB$ rispettivamente in $D$, $E$ e $F$. $BI$ interseca $EF$ in $K$.
    
    Mostrare che $BK\perp CK$. 
    \item Sia $ABC$ un triangolo la cui circonferenza inscritta, di centro $I$, tange $BC$,$CA$ e $AB$ in $D$,$E$ e $F$  rispettivamente. Siano $N$ l'intersezione di $ID$ con $EF$ e $M$ il punto medio di $BC$.
    
    Mostrare che $A$, $N$ e $M$ sono allineati.
\end{enumerate}
\clearpage
\subsection{GM - 3, Esercizi}
\begin{enumerate}
	\item Sia $ABCD$ un quadrilatero ciclico di circocentro $O$. Le rette $AB$ e $CD$ si intersecano in $E$, le rette $AD$ e $BC$ si intersecano in $F$ e le rette $AC$ e $BD$ si intersecano in $P$. Sia $K$ l'intersezione di $EP$ e $AD$ e $M$ la proiezione di $O$ su $AD$.
	
	Mostrare che $BCMK$ è ciclico. 
	\item \textbf{[Fatti su triangolo con mistilinea]} Sia $ABC$ un triangolo iscritto in una circonferenza $\Gamma$ e sia $\gamma$ la circonferenza tangente ai segmenti $AB$, $AC$ e a $\Gamma$ rispettivamente in $E$, $F$ e $T$. Sia $I$ l'incentro di $ABC$. Sia $M$ il punto medio dell'arco $BC$ che non contiene $A$. Sia $V$ l'intersezione di $AT$ con $EF$. 
	
	Mostrare che:
	\begin{itemize}
		\item $I\in EF$ e $IE=IF$;
		\item $MT$, $EF$ e $BC$ sono concorrenti;
		\item $\angle BVE=\angle CVF$.
	\end{itemize}

	\item \textbf{[Teorema di Sawyama-Thébault]} 
	Sia $ABC$ un triangolo di incentro $I$ e sia $D$ un punto sul lato $BC$. Sia $P$ (rispettivamente $Q$) il centro della circonferenza che tange i segmenti $AD$ e $DC$ (rispettivamente $DB$) e la circonferenza circoscritta ad $ABC$. 
	
	Mostrare che $P$, $Q$ e $I$ sono allineati.
	\item \textbf{[NUSAMO 2015/2016 - 5]}
	Sia $ABC$ un triangolo, $I_A$ l'excentro opposto ad $A$
	e $I$ il suo incentro. Sia $M$ il circocentro del triangolo $BIC$ e sia $G$ la proiezione di $I_A$ su $BC$.
	Sia, infine, $P$ l'intersezione fra la circonferenza circoscritta di $ABC$ e la circonferenza di diametro $AI_A$. 
	
	Mostrare che $M$, $G$ e $P$ sono allineati.
	%Inversione nella circonferenza circoscritta a BIC che ha centro in M
	\item \emph{[Copiato in GB]} Siano $A$, $B$ e $C$ tre punti allineati e supponiamo che $P$ sia un punto qualsiasi del piano distinto dai precedenti 3. 
	
	Mostrare che i circocentri dei triangoli $PAB$, $PAC$, $PBC$ e $P$ sono conciclici.
	\item \emph{[Copiato in GB]} Sia $ABC$ un triangolo con ortocentro $H$ e siano $D$, $E$ e $F$ i piedi delle altezze che cadono sui lati $BC$, $CA$ e $AB$ rispettivamente. Sia $T=EF\cap BC$.
	
	Mostrare che $TH$ è perpendicolare alla mediana condotta da $A$.
    %inversione di centro A e raggio AH\cdot HD. La tesi diventa equivalente a mostrare che la circonferenza per D (immagine di H), per l'intersezione della circoscritta con AEF (immagine di T) e A ha la retta AM come diametro. Questo segue perché in effetti M,T',A,D sono ciclici
    \item \textbf{[Teorema di Feuerbach]}\emph{[Copiato in GB]} Mostrare che la circonferenza di Feuerbach è tangente alla circonferenza inscritta e alle circonferenze exinscritte.
    
    \emph{Suggerimento:} Sia $M$ il punto medio di $BC$ e $D$ e $G$ rispettivamente i punti in cui la circonferenza inscritta e quella ex-inscritta opposta ad $A$ incontrano $BC$. Invertire in $M$ con raggio $MD$.
    %Per prima cosa si nota che il piede della perpendicolare e il piede della bisettrice su BC si scambiano perché MH\cdot MI=MD^2. Inoltre si mostra passando per la circoscritta che la retta immagine della circonferenza dei nove punti fa un angolo di beta - gamma con BC. Dunque la cfr dei nove punti va nella simmetrica della retta BC rispetto alla bisettrice che tange entrambe le circonferenze inscritta ed exinscritta. Inoltre queste due si scambiano
    
    \item La circonferenza inscritta nel triangolo $ABC$ è tangente a $BC$, $CA$ e $AB$ in $M$, $N$ e $P$ rispettivamente. 
    
    Mostrare che il circocentro e l'incentro di $ABC$ sono allineati con l'ortocentro di $MNP$.
    %Inversione nella circonferenza inscritta 
\end{enumerate}

 \clearpage
 
 \section{Problemi Basic}
\subsection{GB - 1, Problemi}
\begin{itemize}
 \item \textbf{EGMO 2013 - 1} Nel triangolo $ABC$, si prolunghi il lato $BC$ dalla parte di $C$ di un segmento $CD$ tale che $CD=BC$. Si prolunghi poi il lato $CA$ dalla parte di $A$ di un segmento $AE$ tale che $AE= 2CA$.Dimostrare che, se $AD=BE$, allora il triangolo $ABC$ è rettangolo
 
 \item \textbf{IMOSL 1998 - 5} Sia $ABC$ un triangolo, $H$ l'ortocentro, $O$ il circocentro e $R$ il raggio della circonferenza circoscritta. Sia $D$ il simmetrico di $A$ rispetto a $BC$, $E$ il simmetrico di $B$ rispetto $AC$ e $F$ il simmetrico di $C$ rispetto $AB$.\\
 Dimostrare che $D,E,F$ sono allineati se e solo se $OH=2R$.

 \begin{sol}
 Complessi con circoscritta = circonferenza unitaria
\end{sol}

 
 
 \item \textbf{IMOSL 2015 - G1} Sia $ABC$ un triangolo acutangolo con ortocentro $H$. Sia $G$ il punto per cui il quadrilatero $ABGH$ risulta un parallelogrammo. Sia $I$ il punto della retta $GH$ per cui
la retta $AC$ biseca il segmento $HI$. Sia $J$ l’ulteriore intersezione tra la retta $AC$ e la
circonferenza circoscritta al triangolo $GCI$.
Dimostrare che $IJ = AH$.

\begin{sol}
 Sia $M=GH\cap AC$, Teorema dei seni su $\triangle IJM$ da $\frac{\sin\alpha}{IJ}=\frac{\sin IMJ}{IJ}=\frac{\sin IJM}{MH}=\frac{\sin IGC}{MH}$ per la ciclicità di $GCIJ$. Teorema dei seni su $\triangle MAH$ da $\frac{\sin\alpha}{AH}=\frac{\sin CJH}{MH}$. Per la tesi basta dimostrare che $\widehat{CGH}=\widehat{CAH}=90-\gamma$, ma $CHG$ è rettangolo e $CH=c\cdot cotg(\gamma)=HG\cdot cotg(\gamma)$.
\end{sol}

\item \textbf{ITA TST 2016 - 1} Sia $ABCD$ un quadrilatero. Supponiamo che esista un punto $P$ interno al quadrilatero tale che $\angle APD = \angle BPC = 90^{\circ}$ e $PA \cdot PD = PB \cdot PC$. Sia $O$ il circocentro di $\triangle CPD$.\\
Dimostrare che la retta $OP$ passa per il punto medio di $AB$.
 
 \end{itemize}

\subsection{GB - 2, Problemi}
 \begin{itemize}
    \item \textbf{IMOSL2013 - G2}  Sia $ABC$ un triangolo, e sia $\omega$ la sua circonferenza circoscritta.  Siano $M$ il punto medio di $AB$, $N$ il punto medio di $AC$, $T$ il punto medio dell’arco $BC$ di $\omega$ che noncontiene $A$. La circonferenza circoscritta al triangolo $AMT$ interseca l’asse di $AC$ in un punto $X$ interno al triangolo $ABC$. La circonferenza circoscritta al triangolo $ANT$ interseca l’asse di $AB$ in un punto $Y$ interno al triangolo $ABC$. Le rette $MN$ e $XY$ si intersecano in $K$.\\
    Dimostrare che $KA=KT$.
    
    \begin{sol}La simmetria rispetto all'asse di $AT$ manda $M$ in $X$ e $N$ in $Y$, quindi $K$ rimane fisso e sta sull'asse.
    \end{sol}
    
    \item \textbf{EGMO 2016 - 4} Due circonferenze aventi lo stesso raggio, $\omega_1$ e $\omega_2$ , si intersecano in due punti distinti $X_1$ and $X_2$ . Si consideri una circonferenza $\omega$ tangente esternamente a $\omega_1$ nel punto $T_1$ e internamente a $\omega_2$ nel punto $T_2$.\\ Si dimostri che il punto d’intersezione fra le rette $X_1T_1$ e $X_2T_2$ giace su $\omega$.
    \begin{sol}
    Inversione in $X_1$
    \end{sol}
    
    \item \textbf{Allenamenti EGMO 2019 - G6}
    Dato il triangolo $\Delta ABC$ consideriamo $\omega_B$ la circonferenza passante per A, B e tangente in A al lato AC e, simmetricamente, $\omega_C$ la circonferenza passante per A, C e tangente in A al lato AB. Sia D il punto di intersezione di $\omega_B$ e $\omega_C$ , e sia E il punto sulla retta AD tale che $AD = DE$.\\
    Dimostrare che E sta sulla circonferenza circoscritta al triangolo $\triangle ABC$.

    \begin{sol}
    invertire in A.
    \end{sol}

    \item \textbf{Senior 2013 TF}
    Sia $ABC$ un triangolo. Sia D l’ulteriore intersezione tra la circonferenza passante per C e tangente alla retta AB in A e la circonferenza passante per B e tangente alla retta AC in A.
    Sia E il punto sulla retta AB (diverso da A) tale che $BA = BE$. Sia F l’ulteriore intersezione tra la retta AC e la circonferenza circoscritta al triangolo $ADE$.\\
    Dimostrare che $AC = AF$.

    \begin{sol}
    Invertire in A.
    \end{sol}

    
    
	\item \textbf{IMOSL2011 - G4} Sia $ABC$ un triangolo acutangolo e $\Gamma$ la sua circonferenza circoscritta. Sia $B_0$ il punto medio di $AC$ e $C_0$ il punto medio di $AB$. Sia $D$ il piede dell'altezza da $A$ su $BC$ e sia $G$ il baricentro di $ABC$. Sia $\omega$ la circonferenza passante per $B_0,C_0$ e tangente a $\Gamma$ in un punto $X\neq A$. 
	
	Dimostrare che $D,X,G$ sono allineati.
	
	\begin{sol}
	\emph{nota: la soluzione proposta è basic difficile/medium facile}
	
	Inversione + simmetria di centro A e raggio $\sqrt(AB*AB_0)$, scambia B e $B_0$, C e $C_0$, manda $D$ nel centro di $\Gamma$ O e $\omega$ in una circonferenza per $B$ e $C$ tangente all'immagine di $\Gamma$, $B_0C_0$, in un punto $Y$. Poiché $BC$ e $B_0C_0$ sono paralleli, $Y$ sta sull'asse di $BC$, quindi $OY$ è perpendicolare a $B_0C_0$. 
	
	Sia $T$ l'intersezione delle tangenti a $\Gamma$ per A,X e di $B_0C_0$, è centro radicale di $\Gamma, \omega$ e la circoscritta a $AB_0C_0$. ATXYO è ciclico, l'immagine sotto inversione è la retta XDY. Ora basta mostrare DY intersecato la mediana $AA_0$ è G, ma $AD$ è il doppio di $XA_0$ e sono paralleli, quindi l'intersezione è proprio G.
	\end{sol}
	


	\end{itemize}
	
	
\subsection{GB - 3, Problemi}
\begin{enumerate}
    \item \textbf{Polish MO 2018 - 5} Sia $ABC$ un triangolo acutangolo con $AB\neq AC$
    e siano $E,F$ i piedi delle altezze su $AC$ e $AB$. La tangente in $A$ alla circoscritta interseca $BC$ in $P$. La retta parallela a $BC$ passante per $A$ interseca $EF$ in $Q$. 
    
    Dimostrare che $PQ$ è perpendicolare alla mediana passante per $A$ del triangolo $ABC$
    
    \begin{sol}Assi radicali swag: 1) La circonferenza degenere di centro $A$, la circoscritta a $AEF$ e a $BCEF$ hanno $Q$ come centro radicale (in quanto sta su $EF$ per le ultime due e $AQ$ tange la circoscritta $AEF$ per le prime due). 2) $PA^2=PB\cdot PC$, quindi P sta sull'asse radicale tra $A$ e la circoscritta a $BCEF$. Dunque $PQ$ è asse radicale delle due circonferenze ed è perpendicolare alla congiungente dei centri, che è $AM$
    \end{sol}

\end{enumerate}


\clearpage

\section{Problemi Medium}
\subsection{GM - 1, Problemi}
\begin{enumerate}
	\item \textbf{[BMO 2009 - 2]} Sia $MN$ una segmento parallelo al lato $BC$ del triangolo $ABC$, con $M$ sul lato $AB$ e $N$ sul lato $AC$. Le rette $BN$ e $CM$ si incontrano in $P$. Le circonferenze circoscritte a $BMP$ e $CNP$ si incontrano in due punti distinti $P$ e $Q$. 
	
	Mostrare che $\angle BAQ = \angle CAP$.
	\item \textbf{[RMM 2012 - 2]} Sia $ABC$ un triangolo non isoscele e siano $D$, $E$ e $F$ rispettivamente i punti medi dei lati $BC$, $CA$ e $AB$. La circonferenza $BCF$ e la retta $BE$ si intersecano nuovamente in $P$ e la circonferenza $ABE$ e la retta $AD$ in $Q$. Le rette $DP$ e $FQ$ si incontrano in $R$. 
	
	Mostrare che il baricentro $G$ del triangolo $ABC$ giace sulla circonferenza circoscritta al triangolo $PQR$.
	\item \textbf{[USAMO 2016 - Day 2 - 2]} Un pentagono equilatero $AMNPQ$ è inscritto in un triangolo $ABC$ in modo che $M\in AB$, $Q\in AC$ e $N,p \in BC$. Sia $S$ l'intersezione di $MN$ e $PQ$ e denotiamo con $l$ la bisettrice di $\angle MSQ$. 
	
	Mostrare che, detto $I$ l'incentro di $ABC$, $OI$ è parallelo a $l$.
	\item \textbf{[IMO 2008 - 6]} Sia $ABCD$ un quadrilatero convesso con $BA \neq BC$. Siano $\omega_1$ e $\omega_2$ le circonferenze inscritte ai triangoli $ABC$ e $ADC$ rispettivamente. Supponiamo che esista una circonferenza $\omega$ tangente alla retta $BA$ oltre A, alla retta $BC$ oltre $C$, alla retta $AD$ e alla retta $CD$.
	
	Mostrare che le tangenti esterne comuni a $\omega_1$ e $\omega_2$ si intersecano su $\omega$.
	
	\item \textbf{[BMO 2015 - 2]} Sia $ABC$ un triangolo scaleno con incentro $I$ e circonferenza circoscritta $\omega$. $AI$, $BI$ e $CI$ intersecano $\omega$ di nuovo nei punti $D$, $E$ e $F$ rispettivamente. Le rette parallele a $BC$, $CA$ e $AB$ condotte da $I$ intersecano $EF$, $DF$ e $DE$ rispettivamente nei punti $K$, $L$ e $M$.
	
	Mostrare che $K$, $L$ e $M$ sono allineati.
	
	\item \textbf{[IMO 2012 - 1]} Dato un triangolo $ABC$, sia $J$ il centro della circonferenza ex-inscritta opposta al vertice $A$, la quale tange $BC$ in $M$ e le rette $AB$ e $AC$ in $K$ e $L$ rispettivamente. Le rette $LM$ e $BJ$ si intersecano in $F$ e le rette $KM$ e $CJ$ si intersecano in $G$. Sia $S$ il punto d'intersezione fra $AF$ e $BC$ e sia $T$ il punto d'intersezione fra $AG$ e $BC$. 
	
	Mostrare che $M$ è il punto medio di $ST$.
	\item \textbf{[IMO SL 2011 - 4]} Sia $ABC$ un triangolo acutangolo scaleno, e sia $\gamma$ la sua circonferenza circoscritta.
	Siano $A_0$ il punto medio di BC, $B_0$ il punto medio di $AC$ e $C_0$ il punto medio di $AB$. Sia
	$D$ il piede dell’altezza uscente da $A$, $D_0$ la proiezione di $A_0$ sulla retta $B_0C_0$ e $G$ il
	baricentro di $ABC$. Sia $\gamma_1$ la circonferenza passante per $B_0$ e $C_0$, e tangente a $\gamma$ in un
	punto $P$ diverso da $A$.
	\begin{itemize}
	\item Dimostrare che la retta $B_0C_0$ e le tangenti a $\gamma$ nei punti $A$ e $P$ sono concorrenti.
	\item Dimostrare che i punti $D_0$, $G$, $D$, e $P$ sono allineati.
	\end{itemize}
	\item \textbf{[USA TST 2012 - December Test - 1]} In un triangolo acutangolo $ABC$ si ha $\angle A<\angle B$ e $\angle A<\angle C$. Sia $P$ un punto variabile su $BC$. I punti $D$ e $E$ giacciono su $AB$ e $AC$ rispettivamente in modo che $BP=PD$ e $CP=PE$.
	
	Mostrare che al variare di $P$ sul segmento $BC$, la circonferenza circoscritta al triangolo $ADE$ passa per un punto fisso oltre $A$.
\end{enumerate}
\clearpage
\subsection{GM - 2, Problemi}
\begin{enumerate}
	\item \textbf{[China NMO 2017 - 2]} Siano $\omega$ e $\Omega$ di centro $I$ e $O$ rispettivamente la circonferenza inscritta e circoscritta a un triangolo acutangolo
	$ABC$. La circonferenza $\omega$ interseca $BC$ in $D$ e le tangenti a $\Omega$ passanti per $B$ e $C$ si intersecano in $L$.
	Siano $AH$ l'altezza condotta da $A$ a $BC$ e $X$ l'intersezione di $AO$ con $BC$. Siano $P$ e $Q$ le 
	intersezioni di $OI$ con $\Omega$.
	
	Mostrare che $PQXH$ è ciclico se e solo se $A,D$ e $L$ sono allineati.
	\item \textbf{[IMO 2014 - 4]} Siano $P$ e $Q$ punti su un segmento $BC$ di un triangolo acutangolo $ABC$ tali che $\angle PAB = \angle BCA$ e $\angle CAQ=\angle ABC$. Siano $M$ e $N$ punti su $AP$ e $AQ$ rispettivamente tali che $P$ è punto medio di $AM$ e $Q$ è punto medio di $AN$.
	
	Mostrare che l'intersezione di $BM$ e $CN$ giace sulla circonferenza circoscritta di $ABC$.
	
	\item \textbf{[Iran TST 2007 - Day 2 - 3]}
	Sia $\omega$ la circonferenza inscritta ad un triangolo $ABC$ che tange $AB$ e $AC$ rispettivamente in $F$ e $E$. Siano $P$ e $Q$ su $AB$ e $AC$ rispettivamente in modo che $PQ$ sia parallelo a $BC$ e tangente ad $\omega$. Siano $T$ l'intersezione di $EF$ con $BC$ e $M$ il punto medio di $PQ$. 
	
	Mostrare che $TM$ tange $\omega$.
	
	\begin{sol}Se $X=AD\cap \omega$, $TX$ tange $\omega$ per quadrilateri armonici. Poi (XDAY)=-1 e proiettando da $T$ su $PQ$ ottengo che l'intersezione di $TX$
	 con $PQ$ è il suo punto medio
	\end{sol}
	
	\item \textbf{[Iran TST 2009 - Day 2 - 3]}
	In un triangolo $ABC$ è inscritta una circonferenza $\omega$ di centro $I$ che interseca i lati $BC$, $CA$ e $AB$ rispettivamente in $D$, $E$ e $F$. Sia $M$ il piede della perpendicolare da $D$ a $EF$. Sia $P$ il punto medio di $DM$ e $H$ l'ortocentro del triangolo $BIC$.
	
	Mostrare che $PH$ biseca $EF$. 
	\item \textbf{[Romania TST 2007 - Day 7 - 2]}	La circonferenza inscritta al triangolo $ABC$ è tangente 
	ad $AB$ e $AC$ in $F$ ed $E$ rispettivamente. Sia $M$ il punto di $BC$ e $N$ l'intersezione di $AM$ con $EF$. La circonferenza di diametro $BC$ interseca $BI$ e $CI$ in $X$ e $Y$ rispettivamente.
	
	Mostrare che $\displaystyle\frac{NX}{NY}=\displaystyle\frac{AC}{AB}$.
	
	\begin{sol}Usa l'esercizio 13 e nota che DXY è simile ad ABC e ID è bisettrice di YDX. Oppure semplicemente formula seni-lati su IXY e un po' di trigonometria
	\end{sol}
	
	\item \textbf{[IMO SL 2007 - G8]}
	Sul lato $AB$ di un quadrilatero convesso $ABCD$ è preso un punto $P$. Sia $\omega$ la circonferenza inscritta al triangolo $CPD$ e sia $I$ il suo centro. Supponiamo che $\omega$ sia tangente alle circonferenze inscritte ai triangoli $APD$ e $BPC$ in $K$ e $L$ rispettivamente. Siano $E$ l'intersezione delle rette $AC$ e $BD$ e $F$ l'intersezione delle rette $AK$ e $BL$.
	
	Mostrare che $E$, $I$ e $F$ sono allineati.
	

\end{enumerate}
\clearpage
\subsection{GM - 3, Problemi}
\begin{enumerate}
	\item \textbf{[USA TST 2007 - 5]} Il triangolo $ABC$ è inscritto in una circonferenza $\Gamma$. Le tangenti a $\Gamma$ condotte da $B$ e $C$ si intersecano in $T$. Il punto $S$ è sulla retta $BC$ dimodoché $AS\perp AT$. Siano $B_1$ e $C_1$ sulla retta $ST$ dimodoché $B_1T=BT=C_1T$.
	
	Mostrare che $ABC$ e $AB_1C_1$ sono simili.
	\item \textbf{[IMO 2005 - 5]}
	Sia $ABCD$ un quadrialtero convesso con $BC=DA$ e $BC$ non parallelo a $DA$. Siano $E$ e $F$ su $BC$ e $DA$ rispettivamente tali che $BE=DF$. Siano $P$ l'intersezione di $AC$ e $BD$, $Q$ l'intersezione di $BD$ e $EF$ e $R$ l'intersezione di $EF$ e $AC$.

	Mostra che, al variare di $E$ e $F$, la circonferenza circoscritta al triangolo $PQR$ passa per un punto fisso (oltre $P$). 
	
	\item \textbf{[?]}  Sia $ABC$ un triangolo e siano $D$ e $E$ i piedi delle altezze relative ad $A$ e $B$, rispettivamente,  le quali siintersecano  in $H$.   Sia $M$ il  punto  medio  di $AB$ e  supponiamo  che  le  circonferenze circoscritte a $ABH$ e $DEM$ si intersechino nei punti $P$ e $Q$ (con $P$ e $A$ sullo stesso lato di $CH$).
	%PreIMO Mattino 4 2016
	
	Mostrare che le rette $PH$ e $MQ$ si incontrano sulla circonferenza circoscritta ad $ABC$. 
	\item \textbf{[IMO SL 2006 - 9]}
	Sui lati $BC$, $CA$ e $AB$ di un triangolo $ABC$ si scelgano tre punti $A_1$, $B_1$ e $C_1$ rispettivamente. Le circonferenze circoscritte a $AB_1C_1$, $BC_1A_1$ e $CA_1B_1$ intersecano la circonferenza circoscritta ad $ABC$ in $A_2$, $B_2$ e $C_3$ rispettivamente. Siano, inoltre, $A_3$, $B_3$ e $C_3$ rispettivamente i simmetrici di $A_1$, $B_1$ e $C_1$ rispetto ai punti medi dei lati del triangolo su cui giacciono. 
	
	Mostrare che i triangoli $A_2B_2C_2$ e $A_3B_3C_3$ sono simili.
	
	\begin{sol}$A_2$ è il centro della spilar similiarity che porta $BC_1$ in $CB_1$ quindi $A_2C/A_2B=B_1C/C_1B=AB_3/AC_3$ da cui $A_2BC$ è
	simile ad $AC_3B_3$ e da qui sono angoli 
	\end{sol}
	\item \textbf{[EGMO 2013 - 5]}
	Sia $\Omega$ la circonferenza circoscritta ad un triangolo $ABC$. La circonferenza $\omega$ è tangente ai lati $AC$ e $BC$ e internamente alla circonferenza $\Omega$ in un punto $P$. Una retta parallela ad $AB$ che interseca l'interno del triangolo $ABC$ è tangente a $\omega$ in $Q$.
	
	Mostrare che $\angle ACP = \angle QCB$.
	\item \textbf{[IMO SL 2003 - 4]}
	 Siano  $\Gamma_1$, $\Gamma_2$, $\Gamma_3$, $\Gamma_4$ 
	 circonferenze distinte tali che
	 $\Gamma_1$ e $\Gamma_3$ (così come $\Gamma_2$ e $\Gamma_4$) siano tangenti esternamente in $P$. Supponiamo che $\Gamma_1$ e $\Gamma_2$, $\Gamma_2$ e $\Gamma_3$, $\Gamma_3$ e $\Gamma_4$, $\Gamma_4$ e $\Gamma_1$ si intersechino in $A$, $B$, $C$ e $D$ rispettivamente e che nessuno di questi punti sia $P$.
	 
	 Mostrare che 
	 $$
	 \frac{AB\cdot BC}{AD\cdot DC}=\frac{PB^2}{PD^2} .
	 $$
	 
	 \item \textbf{[IMO 2015 - 3]} Sia $ABC$ un triangolo acutangolo con $AB > AC$. Sia $\Gamma$ la sua circonferenza circoscritta, $H$ il suo ortocentro, e $F$ il piede dell'altezza condotta da $A$. Sia $M$ il punto medo di $BC$. Sia $Q$ il punto su $\Gamma$ tale che $\angle HQA = 90^{\circ}$ e sia $K$ il punto su $\Gamma$ tale che $\angle HKQ = 90^{\circ}$. Assumiamo che $A$, $B$, $C$, $K$ e $Q$ sono tutti distinti e giacciono su $\Gamma$ in quest'ordine. 
	 
	 Mostrare che le circonferenze circoscritte ai triangoli $KQH$ e $FKM$ sono fra loro tangenti.
	 \begin{sol}Inversione di centro H che fissa la circonferenza circoscritta ad ABC. K'Q' diviene perpendicolare ad AK' che è l'asse di F'M' e dunque K'Q' è la tangente a K' nella circonferenza circoscritta a F'M'K'.
	 \end{sol}

		
\end{enumerate}

 \clearpage 

 \begin{thebibliography}{9}
	\bibitem[1]{} Gunmay Anda, \emph{Inversion on the fly}, \url{http://services.artofproblemsolving.com/download.php?id=YXR0YWNobWVudHMvNy85LzRiMmFiYzk1NTgxNjQyZGNhNjEzZDkxOGQ0OTFmN2UyYWFlMDc3LnBkZg==&rn=SW52ZXJzaW9uLnBkZg==}
	\bibitem[2]{} Du\v{s}an Djuki\' c, \emph{Inversion}, \url{http://memo.szolda.hu/feladatok/inversion_ddj.pdf}
	\bibitem[3]{} Paul Yiu, \emph{Introduction to the geometry of the triangle}, \url{http://math.fau.edu/yiu/YIUIntroductionToTriangleGeometry121226.pdf}
	\bibitem[4]{} Marko Radovanovi\'c, \emph{Complex numbers in geometry}, 
	\url{https://www.google.com/url?sa=t&rct=j&q=&esrc=s&source=web&cd=1&ved=2ahUKEwjZiqjuhLDkAhUN16QKHZTNBnAQFjAAegQIAhAC&url=http\%3A\%2F\%2Fservices.artofproblemsolving.com\%2Fdownload.php\%3Fid\%3DYXR0YWNobWVudHMvOS9iLzZhNGM2M2Y0NzZiNGY3MWE3ZTI0ZTRiY2Y4OGIwMzhiN2IyNzFhLnBkZg\%3D\%3D\%26rn\%3DbWFya28tcmFkb3Zhbm92aWMtY29tcGxleC1udW1iZXJzLWluLWdlb21ldHJ5LnBkZg\%3D\%3D&usg=AFQjCNFBeoyb2eMJWQnC3Q7qMMS3okG1Kw}
	
	%\url{http\%3A\%2F\%2Fservices.artofproblemsolving.com\%2Fdownload.php\%3Fid\%3DYXR0YWNobWVudHMvOS9iLzZhNGM2M2Y0NzZiNGY3MWE3ZTI0ZTRiY2Y4OGIwMzhiN2IyNzFhLnBkZg\%3D\%3D\%26rn\%3DbWFya28tcmFkb3Zhbm92aWMtY29tcGxleC1udW1iZXJzLWluLWdlb21ldHJ5LnBkZg\%3D\%3D&usg=AFQjCNFBeoyb2eMJWQnC3Q7qMMS3okG1Kw}
	\bibitem[5]{} Milivoje Luki\'c, \emph{Projective geometry}, \url{http://memo.szolda.hu/feladatok/projg_ml.pdf}
	\bibitem[6]{} Ercole Suppa, \emph{Divisione armonica}, \url{http://www.dma.unifi.it/~mugelli/Incontri_Olimpici_2010/19-Geometria-Testi/02-Ercole_Suppa-Divisione_Armonica.pdf}
	\bibitem[7]{} Yufei Zaho, \emph{Cyclic quadrilaterals -- the big picture}, \url{http://yufeizhao.com/olympiad/cyclic_quad.pdf}
	\bibitem[8]{} Yufei Zaho, \emph{Circles}, \url{http://yufeizhao.com/olympiad/imo2008/zhao-circles.pdf}
	\bibitem[9]{} \emph{Mathlinks}, \url{https://artofproblemsolving.com/wiki/index.php?title=MathLinks}
	\bibitem[10]{napoleoncomplex}Cut the Knot \emph{Napoleon Theorem - Proof with complex numbers} \url{ https://www.cut-the-knot.org/proofs/napoleon_complex2.shtml}
\end{thebibliography}
 
 
\end{document}
