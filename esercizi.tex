
\section{Esercizi Basic}
\subsection{GB-1, Esercizi}
\begin{enumerate}
\item Teoremi di Napoleone e Vecten (enunciato in G2)

\end{enumerate}


\subsection{GB-3, Esercizi}
\begin{enumerate}

	\item \emph{[Copiato da GM]} Sia $ABC$ un triangolo con ortocentro $H$ e siano $D$, $E$ e $F$ i piedi delle altezze che cadono sui lati $BC$, $CA$ e $AB$ rispettivamente. Sia $T=EF\cap BC$.
	
	Mostrare che $TH$ è perpendicolare alla mediana condotta da $A$.
	
	%Oltre alla soluzione per inversione, pensavo anche qualcosa con gli assi radicali: il centro radicale delle circonferenze per AEFH, BCH, ABEF è T, quindi TH passa per l'intersezione di AEFH e BCH che chiamo P. Allora APH è retto in quanto diametro


    %inversione di centro A e raggio AH\cdot HD. La tesi diventa equivalente a mostrare che la circonferenza per D (immagine di H), per l'intersezione della circoscritta con AEF (immagine di T) e A ha la retta AM come diametro. Questo segue perché in effetti M,T',A,D sono ciclici
	
	\item Sia $ABC$ un triangolo, $E,F$ i piedi delle altezze su $AC,AB$. Sia $H$ l'ortocentro, $M$ il punto medio di $BC$ e $Q$ l'intersezione più vicina ad $A$ di $HM$ con la circoscritta $\Gamma$. Sia $T=EF\cap BC$. Dimostrare che $T,Q,A$ sono allineati.
	
	% Usare due fatti 1) Il punto Q è l'intersezione di $\Gamma$ con la circonferenza di diametro $AH$ ed è allineato con H,M e il simmetrico di A rispetto O 2) Assi radicali di AQH, ABC, BCEF.
	
	\item Sia $ABC$ un triangolo con $I$ incentro e $I_A$ centro della circonferenza ex-inscritta relativa ad $A$. Sia $\Gamma$ la circonferenza circoscritta ad $ABC$ e sia $M$ il punto medio dell'arco $BC$ non contenente $A$.
	
	Dimostrare che $B,I,C,I_A$ si trovano su una stessa circonferenza di centro $M$
	
	\item Sia $ABC$ un triangolo, $\omega$ la circonferenza inscritta tangente a $BC$ in $D$. Sia $M$ il punto medio di $BC$ e $E$ il simmetrico di $D$ rispetto a $M$. Sia $T$ il diametralmente opposto a $D$ in $\omega$. Dimostrare che $A,T,E$ sono allineati.
	
	%Calcolare i segmenti di tangenza di inscritta ed ex-inscritta, poi omotetia in $A$
	
	\item Proprietà varie della circonferenza di Feuerbach. 
	\end{enumerate}
